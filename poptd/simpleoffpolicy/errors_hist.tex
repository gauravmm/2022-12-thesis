% \begin{figure}[t]
%     \centering
%     \input{poptd/simpleoffpolicy/errors.tex}
%     % \vspace{-1.75em}
%     \caption{The distribution of Q-function error for naive and POP Q-Learning applied to Figure~\ref{fig:sopmap} over 100 random initializations. POP-Q learns the Q-function with substantially less error. }
%     \label{fig:soperr}
% \end{figure}

\begin{tikzpicture}
    \begin{groupplot}[
            group style={
                    % set how the plots should be organized
                    group size=1 by 2,
                    % only show ticklabels and axis labels on the bottom
                    x descriptions at=edge bottom,
                    % set the `vertical sep' to zero
                    vertical sep=0pt,
                },
            width=1.1\columnwidth,height=.4\columnwidth,
            ybar,
            ymin=0, ymax=0.6,
            yticklabel=\empty,
            xmin=-2, xmax=7,
        ]

        % Naive
        \nextgroupplot[
            ylabel=Naive Q,
            xlabel=Error,
        ]

        \addplot +[hist={bins=18,density,data min = -2,data max = 7,},color=naivetds,draw=naivetd] table [y index=2] {poptd/simpleoffpolicy/qval_err.dat};

        \node at (axis cs:-1.22,0.6) [anchor=north west] {On-Policy};
        % \node at (axis cs:6.12,0.8) [anchor=north west] {$\|V\|_2$};

        \draw[dotted,thick] (axis cs:-1.22,0.) -- (axis cs:-1.22,1.6);
        % \draw[dotted,thick] (axis cs:6.12,0.) -- (axis cs:6.12,1.6);

        % POP-Q-Learning
        \nextgroupplot[
            xlabel=log Error,
            ylabel=POP-Q,
        ]
        \addplot +[hist={bins=18,density,data min = -2,data max = 7,},color=poptds,draw=poptd] table [y index=1] {poptd/simpleoffpolicy/qval_err.dat};

        \draw[dotted,thick] (axis cs:-1.22,0.) -- (axis cs:-1.22,1.6);
        % \draw[dotted,thick] (axis cs:6.12,0.) -- (axis cs:6.12,1.6);
    \end{groupplot}

\end{tikzpicture}
